\mychapter{Timescales}

\mysection{Introduction}

It might first seem strange to talk about timescales, the obvious belief is that there is only one timescale and we seasure events on it. Sadly, this is not the case and some errors might occur due to this belief. In this chapter, we will briefly describe the different timescales, their relation, and their impact on calculations.

\mysubsection{Impact of the error on time}

As all our numerical data is in the second of arc unit, we need to first make some calculations that will allow conversion from an error on time (expressed in seconds) in an error of angle, expressed in seconds of arc. Of course, this conversion depends on too many things to get even an approximation of the function giving this conversion. But what's interesting for us is the maximum conversion rate, meaning the conversion rate in the case where it will induce the maximal anglar error. 

As we are interested in majoring this conversion rate, we will also use many pessimistic approximations. The data we'll use for celestial bodies correspond to the following case:

\begin{itemize}
\item the celestial bodies are at their closest position from the Earth
\item the celestial bodies are at their maximum speed (even the Earth)
\item the observation takes place at the top of Mt Everest, on the Ecliptic, when the observer is closest to the celestial body
\end{itemize}

As we consider short times, we'll make the following approximations:
\begin{itemize}
\item the planet goes in a linear direction
\item due to the rotation of the Earth on itself, the observer goes in the opposite linear direction
\item due to the rotation of the Earth around the Sun, the observer goes in the same linear direction as the one just before
\end{itemize}

A few notations:
\begin{itemize}
\item $r_{min}$ is the distance between the observer and the center of the planet, which we'll consider constant because of the short times
\item $v_{p}$ is the maximum linear speed of the planet
\item $v_{obs}=v_{E}+v_{S}$ is the maximum linear speed of the observer, the sum of the maximum linear speed of the observer due to the rotation of the Earth on itself and around the Sun. For the Sun and the Moon, we only have $v_{obs}=v_{E}$.
\end{itemize}

We'll express distances in kilometers, times in seconds and speeds in kilometers by second.

A first thing can be to determine $v_{obs}=v_E+v_S$. If we consider the observer as we defined it, and say Earth is round, in a 24h day, he'll travel $2\pi\times r_{E}$ where $r_{E}=R_\oplus+alt._{obs}=6375.9km$ is the distance between him and the geocenter. We have thus 

\begin{equation}
v_E = \frac{2\pi r_E}{86400} = 0.46km/s
\end{equation}

Now for $v_S$, we easily major it by $v_S<110000km/h=3.96\time 10^8km/s$\footnote{reference missing}. For planets, we can thus neglect $v_E$ and major $v_{obs}$:

\begin{equation}
v_{obs} < 4\times 10^8km/s
\end{equation}

Now, the maximum error angle by second is the difference between the angle and the angle one second later. It has its maximum when the first angle is 0, so we just need to calculate the angle after one second. This angle corresponds to Fig~\ref{maxangularerror}.

\begin{figure}
\centering 
\begin{tikzpicture}[scale=2]
\draw [<-] (-2,0) node[anchor=north] {$P_{obs}(t_1)$} -- node[anchor=north] {$\vec{v}_{obs}$} (0,0) node[anchor=north] {$P_{obs}(t_0)$} -- %node[northwest=2mm] {$r_{min}$} 
(0,2) node[anchor=south] {$P_{cb}(t_0)$};
\node [anchor=center] at (-0.25,1.35) {$r_{min}$};
\draw [->] (0,2) -- node[anchor=south] {$\vec{v}_{cb}$} (2,2) node[anchor=south] {$P_{cb}(t_1)$};
\draw [gray] (2,2) -- (-2,0);
\draw (-1.3,0) arc (0:26.56:0.7);
\node [anchor=center] at (-1,0.2) {$\Theta_{max}$};
\end{tikzpicture}
\caption{Maximum angular error for short times}\label{maxangularerror}
\end{figure}

This figure can be easily simplified in a simple triangle and we can deduce 

\begin{equation}
\Theta_{max}=atan(\frac{v_{obs}+v_{cb}}{r_{min}})
\end{equation}

The values used in the following calculations are the ones given in Table~\ref{table:planetvalues}\footnote{These values come from random Internet sites consistent with each other, but if someone has a more reliable source, I'd be glad to use it!}.

\begin{table}
\centering
\begin{tabular}{|l|S[table-format=1.2,table-figures-exponent = 1|S[table-format=1.1,table-figures-exponent = 1|}
\hline
\multicolumn{1}{|c|}{\textbf{Celestial body}} & \multicolumn{1}{c|}{\textbf{$r_{min}$ (km)}} & \multicolumn{1}{c|}{\textbf{$v_{cb}$ (km/s)}} \\\hline
Sun & 1.47e8 & 1.0e8\\\hline % min: 147Mkm, mean: 150km
Moon & 3.56e5 & 1706\\\hline % min: 360kkm, mean: 385kkm
Mercury & 7.7e7 & 7\\\hline %  min: 77 Mkm, mean: 100Mkm
Mars & 5.5e7 & 3\\\hline % min: 54.6Mm, mean: 225Mkm
Venus & 3.8e7 & 4\\\hline % min: 38Mkm, mean: 150Mkm
Saturn & 1.2e9 & 453\\\hline % min  1200Mkm, mean: 1450Mkm
Jupiter & 6.3e11 & 8e-4 \\\hline % min: 629kMkm, mean: 770kMkm
\end{tabular}
\caption{Planet caracteristics used in the calculations}
\label{table:planetvalues}
\end{table}

\mysection{Astronomical timescales}

\mysubsection{Atomic time (TAI)}

\mysubsection{Earth Rotation Time}

\mysubsection{Dynamical Time}

\mysubsubsection{Based on TAI}

\mysubsubsection{Truely relativistic time}

\mysection{Time used in calendrical calculations}

\mysubsection{Mean Sun?}

\mysection{Conversions}
