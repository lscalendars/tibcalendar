\documentclass[%
a4paper,% paper size
pagesize,%
12pt,% text size
parskip=off,% inter-paragraph space, automatically sets paragraph indentation
bibliography=totoc,% bibliography in the table or contents
numbers=noenddot,%
DIV=12,% margin settings
twoside=semi,% right and left difference, but same margins
normalheadings%headings=medium,% small headings
]{scrbook}

\usepackage{eroux}

\title{Astronomical calculations in lunisolar calendars}
\subtitle{Overview of accuracy and optimization possibility}
\author{Élie Roux \href{mailto:elie.roux@telecom-bretagne.eu}{<elie.roux@telecom-bretagne.eu>}} 
\date{\today} 

\pagestyle{scrheadings}
\ifoot[]{} 
\cehead[]{\textsc{\rightmark}}
\cohead[]{\textsc{Astronomical calculations in lunisolar calendars}}
% et enfin le numéro de page dans le pied de page extérieur
\ofoot[\oldstylenums\thepage]{\oldstylenums\thepage}
\cfoot[]{}
\ohead[]{}
\ihead[]{}

\let\mychapter\chapter
\let\mysection\section
\let\mysubsection\subsection
\let\mysubsubsection\subsubsection
\let\mysubsubsubsection\subsubsubsection

\def\TODO{}

\begin{document}

\maketitle

\tableofcontents
%\cleardoubleevenemptypage
%\clearpage
\newpage

\mychapter{Introduction}

\mysection{What it this document?}

This article aims at giving an overview of the main inaccuracies encounterable in astronomical calculations when used in calendrical calculations.

\mysection{Genesis}

Working on Tibetan calendar calculations, I realized that there was no documentation on how to use the main astronomical calculations (such as VSOP87) to get the data necessary to build a calendar. These include the different approximations we would naturally make with the result, the induced discrepancy of which I have never seen documented.

I guess many people did these calculations, such as the authors of the main astronomical libraries, but no documentation is available on the global accuracy of their calculations nor on why they would take into account a phenomenon or not.

Instead of redoing these calculations myself, I decided to compute them fully and document them, in order:
\begin{itemize}
\item to make the accuracy gain of some considerations documented
\item to analyze and document the precision of the calculations used in my astronomical library\TODO
\end{itemize}

This article is made from the point of view of calendrical calculations but will be meaningful with any calculation from the point of view of a human observer on the Earth.

This article will not give a very precise average error for all phenomena, but:
\begin{itemize}
\item the maximum error
\item an error computed with what seems average values
\item a conclusion on the necessity to take the phenomenon into account according to the desired precision
\end{itemize}



\mysection{Conventions used in this article}

\mysubsection{Assertions}

In this article, we will consider that latest JPL ephemeris \TODO are fully accurate. 

We consider that all computer calculations are approximated as defined by IEEE 754/784, so calculations are supposed to happen in C99 or Fortran 2003. If you use a non-prehistoric processor and compiler, it should be the case though.

\mysubsection{Notation and vocabulary}

\mysubsubsection{Trueness and precision}

This is quite important here to differenciate trueness and precision, both being part of accuracy. These terms are officially defined in \cite{VIM} as closeness of agreement between:

\begin{description}
\item[trueness] the average of an infinite number of replicate measured quantity values and a reference quantity value
\item[precision] indications or measured quantity values obtained by replicate measurements on the same or similar objects under specified conditions
\item[accuracy] a measured quantity value and a true quantity value of a measurand
\end{description}

It makes sense to translate them to astronomical calculations, taking a calculation as a \emph{measure}. Figure~\ref{ATP} is useful to get a better understanding.

\begin{figure}[h]
\centering
\def\svgwidth{10cm}
\input{ATP.pdf_tex}
%\caption{Accuracy, Precision and Trueness of calculations\\This image comes from Wikipedia and is under the same license as this document\TODO}
\label{ATP}
\end{figure}

\mysubsection{Astronomical values}

\mysubsubsection{Orbital caracteristics}

For future calculations, we'll need a set of orbital values which we'll present and justify here. The values we'll use for $a$ and $e$, presented in Table \cite{table:planetorbitalvalues}, come from \cite{NASA-factsheet}. % (TODO: http://nssdc.gsfc.nasa.gov/planetary/planetfact.html)

\begin{table}
\centering
\begin{tabular}{|l|S[table-format=1.3,table-figures-exponent=1]|S[table-format=1.2,table-figures-exponent=1]|}
\hline
\multicolumn{1}{|c|}{\textbf{Celestial body}} & \multicolumn{1}{c|}{\textbf{$a$} (km)} & \multicolumn{1}{c|}{\textbf{$e$}} \\\hline
Sun (Earth) & 1.496e8 & 1.67e-2\\\hline
Moon & 3.84e5 & 5.49e-2\\\hline
Mercury & 5.79e7 & 2.056e-1\\\hline
Mars & 2.279e8 & 9.35e-2\\\hline
Venus & 1.082e8 & 6.77e-3\\\hline
Saturn & 1.433e9 & 5.65e-2\\\hline
Jupiter & 7.786e8 & 4.89e-2 \\\hline
\end{tabular}
\caption{Planet caracteristics used in the calculations}
\label{table:planetorbitalvalues}
\end{table}

Then we need the maximum speed of celestial bodies. The first estimation of it (which we'll use) is obtained by the Kepler laws: $$\sqrt{2GM}\times \sqrt{\frac{1}{r}-\frac{1}{2a}}$$. We use $G=6.67384\times 10^{2} km^3kg^{-1}s^{-2}$\footnote{As recommended by \cite{CODATA}}, $GM_{Sun}=1.3271244\times 10^{11} km^3s^{-2}$ and $GM_{Earth}=3.986004\times 10^{5}km^3s^{-2}$ (as recommended by TODO). The maximum speed of a celestial body is thus $v_{max}=\sqrt{\frac{GM}{a}}\times \sqrt{\frac{1+e}{1-e}}$. Expressed numerically, we have:

% TODO: http://maia.usno.navy.mil/NSFA/NSFA_cbe.html\#ConstGrav2009

\begin{itemize}
\item for planets, $v_{max} \approx{} \frac{364297.18}{\sqrt{a}}\times \sqrt{\frac{1+e}{1-e}}$\
\item for the Moon, $v_{max} \approx{} \frac{631.34}{\sqrt{a}}\times \sqrt{\frac{1+e}{1-e}}$
\end{itemize}

These formulas correspond to the formulas page 238 of \cite{Meeus}, except that the latter have $a$ in astronomical unit. A few calculations will require minimum speed of the plantets, for it we simply take the last formulas, replacing $e$ by $-e$.

We will also need the shortest possible distance from Earth to another celestial body. For the Sun and Moon this is straightforward; for other planets we obtain a minimal bound by substracting the sun distance of the Earth at Aphelion and of the planet at Perihelion (for outer planets, or the opposite for inner planets). This gives the following formulas:

\begin{itemize}
\item for the Sun, $r_{min}=a_{Earth}(1-e_{Earth})$
\item for the Moon, $r_{min}=a_{Moon}(1-e_{Moon})$
\item for inner planets, $r_{min}=a_{Earth}(1-e_{Earth}) - a_{p}(1+e_{p})$
\item for outer planets, $r_{min}=a_{p}(1-e_{p}) - a_{Earth}(1+e_{Earth})$
\end{itemize}

Note that these formulas are not precise as they rely simply on a very theoretical motion, For planets, we'll use the data given in \cite{NASA-factsheet} which are more precise. We can now compute the values of Table \cite{table:planetvalues}. We do the same for maximum distance $r_{max}$.

\begin{table}
\centering
\begin{tabular}{|l|S[table-format=1.2,table-figures-exponent=1]|S[table-format=1.2,table-figures-exponent=1]|S[table-format=2.2,table-figures-exponent=0]|S[table-format=2.2,table-figures-exponent=0]|}
\hline
\multicolumn{1}{|c|}{\textbf{Celestial body}} & \multicolumn{1}{c|}{\textbf{$r_{min}$} (km)} & \multicolumn{1}{c|}{\textbf{$r_{max}$} (km)} & \multicolumn{1}{c|}{\textbf{$v_{max}$} (km/s)} & \multicolumn{1}{c|}{\textbf{$v_{min}$} (km/s)} \\\hline
Sun & 1.47e8 & 1.52e8 & 30.29 & 29.29\\\hline % mean: 150km
Moon & 3.57e5 & 4.07e5 & 1.08 & 0.96\\\hline % mean: 385kkm
Mercury & 7.7e7 & 221.9 & 58.98 & 38.86\\\hline % mean: 100Mkm
Mars & 5.5e7 & 4.01e8 & 26.5 & 21.97\\\hline % mean: 225Mkm
Venus & 3.8e7 & 2.61e8 & 35.26 & 34.78\\\hline % mean: 150Mkm
Saturn & 1.2e9 & 1.66e9 & 10.18 & 9.09\\\hline % mean: 1450Mkm
Jupiter & 5.9e8 & 9.68e8 & 13.71 & 12.43\\\hline % mean: 770kMkm
\end{tabular}
\caption{Planet caracteristics used in the calculations}
\label{table:planetvalues}
\end{table}

\mysubsubsection{Earth caracteristics}

We will consider that Earth is an Ellipsoid with semi-major axis $a=6378.14km$ and flattening $f=298.257$\footnote{These values were adopted by the IAU (\cite[Meeus]) and are the same as WGS84, used in the GPS system.}.

For further calculations, we'll need the maximum radius of the Earth. With the previous data, we get $R_\oplus=6378.14 km$.

% http://en.wikipedia.org/wiki/IERS_Reference_Meridian
% 

\mysection{Thanks}

I would like to thank a lot people without who this paper would never have been possible.

First of course my masters who would certainly like to remain anonymous, for the inspiration to do things well.

Also to Edward Henning, for all the work he's been doing on the Tibetan Calendar, his very precious advices and great disponibility.

Venerable Dr. Knuth provided both many theoretical advances in the field and the most precious tools (\TeX ) that allowed to build this document.

My gratitude goes also to Mathieu Chevrier for his wide knowledge on floating point numbers.

\mysection{License}

This work is under the Creative Commons Attribution-Share Alike 3.0 Unported License. See \cite{CCASA} for details and the complete license.


\mychapter{What precision for calendrical calculations?}

\mysection{Scales used in lunisolar calendars}

\mysubsection{The tibetan calendar}

\mysubsubsection{Angular scales}

A typical Tibetan almanach contains angle data written in the following form $x;y;z$ where $x$ is the lunar mansion (27\textsuperscript{th} of circle), $y$ is in nâdis (60\textsuperscript{th} of lunar mansion) and $z$ in pâlas (60\textsuperscript{th} of nâdi). It is almost the same as Western angle notation, but dividing the circle first in 27, not in 360. So a first approach would say that computations must be accurate enough to get these numbers right. In order to get these right, the precision must be at least $0.5 pâla = 6.67 s$\footnote{$1 pâla$ being a $27\times60\times60^{th}$ of circle and $1s$ being a $360\times60\times60^{th}$ of circle, we have $1 pâla = \frac{360}{27} s = 13.33 s$}.

Internally, calculations are made with more precision, adding two more sets of digits, $x;y;z$ become $x;y;z;z_2;z_3$ where $z_2$ is in shvâsa (6\textsuperscript{th} of pâla) and $z_3$ in bhâga, a fraction of shvâsa that varies according to calculation type. They are mostly:
\begin{itemize}
\item 707, used for Sun and Moon, corresponding to a precision of \num{3.14e-3}\,s
\item 149209, used for planets, corresponding to a precision of \num{1.49e-5}\,s
\end{itemize}

\mysubsubsection{Time scale}

Times is expressed almost in the same way as angles, in the form $x;y;z$ where $x$ is the day of the week (0 is saturday), $y$ is in nâdis (60\textsuperscript{th} of day, 24 minutes) and $z$ in pâlas (60th of nâdi, 24s).

Subunits used internally are shvâsa (6\textsuperscript{th} of pâla, 4s) and the same kind of fraction of shvâsa as angle scales.

\mysubsection{Other calendars}


\mychapter{Precision of different computer methods}

In the process of improving precision in calculations, it is very important to understand the limits of the underlying hardware and software, and the cost of going over them.

\mysection{Different computer representations}

\mysubsection{General representation}

Real numbers are represented by floating point numbers, that can be schematized as:
\begin{itemize}
\item a sign, encoded on 1 bit
\item an exponent $e$, encoded in several bits (11 on double precision)
\item a significand $s$, encoded in the rest of the bits (52 on double precision), these bits are called the mantissa
\end{itemize}

In arbitrary precision libraries, operations are made in software and thus doesn't depend too much on hardware representation, at the cost of an important speed loss. The system is the same though, but the size of the exponent and the mantissa is given by the user\footnote{Note that the mainstream GMP/MPFR doesn't allow to set truely arbitrary exponent}.

The representation size of a floating point number vary according to hardware architecture and user input. We will take into consideration here the most common types (which should cover 99.99\% of use cases). We describe here the IEEE-754 standard as implemented in C99, but there are strict equivalents in Fortran.

\mysubsection{Norms and definitions}

This paragraph describes the most simple description of floats, namely the description of the main types as in the IEEE-754 norm. This can be summarized in the following table:

\begin{center}
\vspace{\spacearoundtables}
\begin{tabular}{|l|S|S|S|}
\hline
\multicolumn{1}{|c|}{\textbf{name}} & \multicolumn{1}{c|}{\textbf{total}} & \multicolumn{1}{c|}{\textbf{exponent}} & \multicolumn{1}{c|}{\textbf{mantissa}} \\\hline
Single precision & 32 & 8 & 23 \\\hline
Double precision & 64 & 11 & 52 \\\hline
Quadruple precision & 128 & 15 & 112 \\\hline
\end{tabular}
\vspace{\spacearoundtables}
\end{center}

But as we will see, there are some discrepancies between this and reality, due to several implementation choices that we will detail in the next sections.

\mysubsection{Hardware implementation}

There is currently no widespread hardware that can make operations on floating points with more than 80 bits\footnote{though the z/Architecture of IBM Mainstream Servers can compute 128-bit float operations.}, and this limitation tends to be more and more restrictive. This section will explain the different hardware implementations. It is a very important topic to understand as it is non-obvious and implementation choices might seem very strange.

First we can distinguish between two floating point computing hardwares:
\begin{enumerate}
\item[FPU]\footnote{\emph{Floating Point Unit}, or x87.} This is the historical hardware introduced by Intel, present in all x86 and x86\_64 computers. This hardware has 80-bits registers and thus cannot make more precise operations.
\item[SIMD]\footnote{\emph{Single Instruction, Multiple Data}.} Single operations work with at most 64-bit registers and are thus less precise. SIMD offers thought the possibility to make several operations at the same time, hence the trend to prefer it. SSE, AVX, NEON, etc. are all following this architecture. Most complex operations such as trigonometric ones are not implemented and fall back on \emph{FPU}.
\end{enumerate}

Floating point numbers represented in 80-bit registers follow the \emph{extended precision} of IEEE-754 standards and is composed of a 15-bit exponent and a 64-bits mantissa.

Modern x86 and x86\_64 hardware have both SIMD (SSE or AVX) and FPU.

On decent compilers, it is possible to decide between SIMD, FPU or both\footnote{on \texttt{gcc}, the \texttt{-mfpmath} switch allows this, see \url{http://gcc.gnu.org/onlinedocs/gcc/i386-and-x86_002d64-Options.html} and \url{http://gcc.gnu.org/wiki/Math_Optimization_Flags}}. It is thus necessary to study this topic if you want to work on precision. By default, on x86 processors, the FPU is used, while SIMD is the default on x86\_64 processors and ARM processors.

It is important to note that on ARM processors (present on smartphones, tablets, etc.), floating point operations are sometimes done in a VFP, a kind of modified FPU. This VFP has no clear specification, but it seems not to have 80-bit registers, so smartphones and tablets should be considered to have only 64-bit registers.

Another form of harware implementation was present in PowerPC and SPARC processors, where 128-bit numbers could be encoded in two 64-bit registers with operations combining them. This arithmetic had a precision of about 106-bit.

\mysubsection{Software fallback}

It is possible to make floating point arithmetic with software, at the cost of very significant speed reduction. These implementations are of two types:

\begin{enumerate}
\item[\enumstyle{Compiler}] Sometimes languages specifications (or extensions) cannot be implemented with hardware, and thus compilers provide a software fallback for some types
\item[\enumstyle{Libraries}] Some libraries can do arbitrary precision floating point arithmetic. The most famous ones being GMP\footnote{\url{http://gmplib.org/}} and MPFR\footnote{\url{http://www.mpfr.org/}}.
\end{enumerate}

\mysubsection{C implementation}

This part will describe C99, but anyone using a language for astronomical calculations should get knowledge on this topic for the language he chooses.

The discrepancy between hardware and a naive float approach is naturally also present between the naive approach and the various implementations of compilers. Indeed, compilers tend to stick as close to hardware as possible (which is their role) in order to use as few costy software emulation as possible. 

As we've seen, almost no hardware is capable of more than 80-bit calculations, thus hardware 128-bit calculation is impossible. We can also safely assume that users using the C99 keyword \texttt{long double} don't want software emulation. To solve this dilemma, compilers chose different options, the main things to know being summed up in the next paragraphs.

\mysubsubsection{\texttt{double} on FPU}

\texttt{double} keyword always uses 64-bit representation in memory, but when it comes to representation in (80-bit) FPU registers, two behaviours are possible:
\begin{enumerate}
\item[\enumstyle{80-bit mode}] calculations and intermediate values are made in 80-bit, which improves the precision of the calculations\footnote{but may lead in wrong comparaison results, see \url{http://gcc.gnu.org/wiki/x87note}}
\item[\enumstyle{64-bit mode}] the mantissa of intermediate values is rounded to 53 bits (thus giving double precision precision)\footnote{This is what the \texttt{-mpc64} option of gcc or the \texttt{/fp:precise} option of MSVC do, see \url{http://gcc.gnu.org/onlinedocs/gcc/i386-and-x86\_002d64-Options.html} and \url{http://msdn.microsoft.com/en-us/library/e7s85ffb.aspx}.}.
\end{enumerate}

In C99, it is possible to know the intermediate rounding of operations with the C99 macro \texttt{FLT\_EVAL\_METHOD}\footnote{\url{http://pubs.opengroup.org/onlinepubs/009695399/basedefs/math.h.html}}.

\mysubsubsection{\texttt{long double} on FPU}

\texttt{long double} can have several representations:
\begin{enumerate}
\setlength{\itemsep}{0pt}
\setlength{\parskip}{0pt}
\item[\enumstyle{64-bit}] this is the default on MSVC, \texttt{long double} being a synonym of \texttt{double}\footnote{see \url{http://msdn.microsoft.com/en-us/library/9cx8xs15.aspx}}
\item[\enumstyle{96-bit}] this is the default on gcc. In this case, 16 bits are not used in calculations in 80-bit mode.
\item[\enumstyle{128-bit}] this is the case on gcc with the \texttt{-m128bit-long-double}\footnote{see \url{http://gcc.gnu.org/onlinedocs/gcc/i386-and-x86\_002d64-Options.html}}. 48 bits are not used in calculations.
\end{enumerate}

For the sake of completeness, it's important to notice here that gcc allows \texttt{\_\_float80}\footnote{\url{http://gcc.gnu.org/onlinedocs/gcc/Floating-Types.html}} type which is a synonym of \texttt{long double} on x86 and x86\_64 architectures.

\mysubsubsection{\texttt{long double} on SIMD and ARM}

The x86\_64 ABI\footnote{\url{http://www.x86-64.org/documentation/abi.pdf}} states that \texttt{long double} has intermediate values and calculations is in extended precision (80-bit), all operations being performed on the FPU; so no SIMD code will be used with the keyword \texttt{long double}.

The ARM Development Tools\footnote{\url{http://infocenter.arm.com/help/index.jsp?topic=/com.arm.doc.dui0067d/BABFCGFC.html}} state that \texttt{long double} is 64-bit long, and thus a synonym of \texttt{double}.

\mysubsubsection{Summing up}

We can thus construct the following table summing up the previous paragraphs, giving some name conventions we'll use later. These names are just convenient and (though largely the same) not identical to IEEE-754 standard. The number of bits cells are in the form $x$/$y$, where $x$ is the number of bits used in memory and $y$ the number of bits used in calculations and intermediate values.

\begin{table}[h]
\begin{tabu} to \linewidth{|X[l]|X[c]|X[c]|X[c]|}
\hline
\rowfont[c]{\bfseries} Compiler & SIMD & x86 FPU & x86\_64 FPU
\\\hline
\textbf{GCC} & 32/32 & 32/80 & 32/80\\\hline
\textbf{MSVC} & 32/32 & 32/80 & 32/80\\\hline 
\end{tabu}
\caption{\texttt{float} representation}
\end{table}

\begin{table}[h]
\begin{tabu} to \linewidth{|X[l]|X[c]|X[c]|X[c]|}
\hline
\rowfont[c]{\bfseries} Compiler & SIMD & x86 FPU & x86\_64 FPU
\\\hline
\textbf{GCC} & 64/64 & 64/80 & 64/80\\\hline
\textbf{MSVC} & 64/64 & 64/80 & 64/80\\\hline
\end{tabu}
\caption{\texttt{double} representation}
\end{table}

\begin{table}[h]
\begin{tabu} to \linewidth{|X[l]|X[c]|X[c]|X[c]|}
\hline
\rowfont[c]{\bfseries} Compiler & SIMD & x86 FPU & x86\_64 FPU
\\\hline
\textbf{GCC} & 64/64 & 96/80 & 128/80\\\hline
\textbf{MSVC} & 64/64 & 64/80 & 64/80\\\hline
\end{tabu}
\caption{\texttt{long double} representation}
\end{table}

%\vspace{\spacearoundtables}

%\begin{tabu} to \linewidth{|X[2]|X[c]|X[4]|}
%\hline
%\rowfont[c]{\bfseries} Name & Nb. of bits & Name in C
%\\\hline
%Simple precision & 32/32 & \texttt{float} in SIMD\\\hline
%Mixed simple precision & 32/80 & \texttt{float} in FPU\\\hline
%Double precision & 64/64 & \texttt{double}, \texttt{long double} on ARM and modern instruction sets (SSE, AVX). If on FPU, must be set in 64-bit mode.\\\hline
%Extended double precision & 80/80 & not available as such (only internal registers) \\\hline
%Mixed double precision & 64/80 & \texttt{double} in FPU, also \texttt{long double} on MSVC \\\hline
%Mixed triple precision & 96/80 & \texttt{long double} on x86 FPU\\\hline
%Quadruple precision & 128/128 & \texttt{\_\_float128} of \texttt{gcc}, software emulation except on a few hardwares\\\hline
%Mixed quadruple precision & 128/80 & \texttt{long double} on x86\_64 FPU\\\hline
%\end{tabu}

%\vspace{\spacearoundtables}

%This table is another view of the same data:

%\vspace{\spacearoundtables}

%\begin{tabu} to \linewidth{|X|X[c]|X[c]|X[c]|X[c]|}
%\hline
%\rowfont[c]{\bfseries} C keyword & FPU (64-bit) & FPU (80-bit) & SIMC & MSVC
%\\\hline
%\texttt{float} & 32/64 & 32/80 & 32/32 & no difference\\\hline
%\texttt{double} & 64/64 & 64/80 & 64/64 & no difference \\\hline
%\texttt{long double} & 96/80 or 128/80 & 96/80 or 128/80 & idem \texttt{double} on ARM, mapped to FPU with others & idem \texttt{double} \\\hline
%\texttt{\_\_float128} & 128/128 & 128/128 or 128/128 & error \\\hline
%\end{tabu}

%\vspace{\spacearoundtables}

\mysection{Common errors due to floating point representations}

This section is an overview of the most common errors due to floating point arithmetic, and of their solution.

\mysubsection{Introduction}

One of the problems of floating point arithmetic is that global formulas are almost inexistant and error for each floating point number manipulation should be calculated by hand, depending on the variable maxima and minima, the chosen float representation, etc.

This section will thus describe only general errors and things to know about floating point manipulation. A good introduction to this topic is \cite{Goldberg}, and \cite{Higham} a more recent and complete book. This section describes the general principles described in these.

\mysubsubsection{Notation}

We will use here analytical notation of floating points numbers we can find very commonly. We will represent a floating point number as being in the form $$d.dd...dd\times\beta^e$$where $d.dd...dd$ is the significand and has p digits, $\beta$ is the base (assumed to be even) and $e$ the exponent.

To make the link with computer representations, we would have:
\begin{itemize}
\item $\beta=2$
\item $p$ equal to the number of bits in the mantissa
\item $d.dd...dd$ the mantissa, with d in base 2, for example 1.100110011001100
\item $e$ the exponent
\end{itemize}

\mysubsubsection{Non-} % communativité

\mysubsection{Polynomial calculations}

Horner's method.

fused multiply–add

\mysubsection{Summations}

Two precision optimizations are possible for sums, one costless and one very costy but very efficient.

\mysubsubsection{Small numbers first}

It is a good practice, for sums of more than two floating point numbers, to start the 

\mysubsubsection{Long sums}

\mysubsection{Errors in transcendental functions}

Transcendental functions (like sin and cos) are not mandatorily exactly rounded.

\mysection{Errors induced by floating point in common astronomical calculations}

\mysubsection{Converting floating point error in angle precision error}

\mysubsection{Errors in common astronomical calculations}

\mysubsubsection{Theorical bounds}

\mysubsubsection{Some measures}



\mychapter{The different calculation methods}


\mysection{Variations Séculaires des Orbites Planétaires (VSOP)}

\mysubsection{vsop87}

The error bound of vsop87 calculations as seen from section 3 of \cite{vsop87}\footnote{Strangely, this doesn't seem to correspond exactly to the values given in \texttt{vsop87.doc} as found on \url{ftp://ftp.imcce.fr/pub/ephem/planets/vsop87/vsop87.doc}.} are, converted in seconds:

\begin{table}[h]
\centering
\sisetup{table-format=1.1, table-figures-exponent = 2}
\begin{tabular}{|l|S|}
\hline
\textbf{Celestial body} & \multicolumn{1}{c|}{\textbf{Maximum error}} \\\hline
\textbf{Earth (Sun)} & 5e-3 \\\hline % 5.1e-3 s = 2.5e-8 rad in vsop87.doc
\textbf{Mercury} & 1e-3 \\\hline % 1.2e-3s = 0.6e-8 rad
\textbf{Venus} & 6e-3 \\\hline % 5.1e-3s = 2.5e-8 rad
\textbf{Mars} & 2.3e-2 \\\hline % 2e-2s = 10e-8 rad
\textbf{Jupiter} & 2e-2 \\\hline % 7.2e-2s = 35e-8 rad
\textbf{Saturn} & 0.1 \\\hline % 1.4e-1s = 70e-8 rad
\end{tabular}
\caption{Minimum trueness of vsop87}
\label{table:vsopprecision}
\end{table}

The guaranteed maximal error is \ang{;;1} in the interval
\begin{itemize}
\item $[0,4000]$ for Jupiter and Saturn
\item $[-2000,6000]$ for the other bodies\footnote{except Neptune and Uranus, but these aren't used in calendars.}
\end{itemize}

\mysubsection{vsop2013}

\mysection{ELP/MPP02 and LEA-406}

\mysection{JPL Ephemeris}


\mychapter{Timescales}

\mysection{Introduction}

It might first seem strange to talk about timescales, the obvious belief is that there is only one timescale and we seasure events on it. Sadly, this is not the case and some errors might occur due to this belief. In this chapter, we will briefly describe the different timescales, their relation, and their impact on calculations.

\mysubsection{Impact of the error on time}

As all our numerical data is in the second of arc unit, we need to first make some calculations that will allow conversion from an error on time (expressed in seconds) in an error of angle, expressed in seconds of arc. Of course, this conversion depends on too many things to get even an approximation of the function giving this conversion. But what's interesting for us is the maximum conversion rate, meaning the conversion rate in the case where it will induce the maximal anglar error. 

As we are interested in majoring this conversion rate, we will also use many pessimistic approximations. The data we'll use for celestial bodies correspond to the following case:

\begin{itemize}
\item the celestial bodies are at their closest position from the Earth
\item the celestial bodies are at their maximum speed (even the Earth)
\item the observation takes place at the top of Mt Everest, on the Ecliptic, when the observer is closest to the celestial body
\end{itemize}

As we consider short times, we'll make the following approximations:
\begin{itemize}
\item the planet goes in a linear direction
\item due to the rotation of the Earth on itself, the observer goes in the opposite linear direction
\item due to the rotation of the Earth around the Sun, the observer goes in the same linear direction as the one just before
\end{itemize}

A few notations:
\begin{itemize}
\item $r_{min}$ is the distance between the observer and the center of the planet, which we'll consider constant because of the short times
\item $v_{p}$ is the maximum linear speed of the planet
\item $v_{obs}=v_{E}+v_{S}$ is the maximum linear speed of the observer, the sum of the maximum linear speed of the observer due to the rotation of the Earth on itself and around the Sun. For the Sun and the Moon, we only have $v_{obs}=v_{E}$.
\end{itemize}

We'll express distances in kilometers, times in seconds and speeds in kilometers by second.

A first thing can be to determine $v_{obs}=v_E+v_S$. If we consider the observer as we defined it, and say Earth is round, in a 24h day, he'll travel $2\pi\times r_{E}$ where $r_{E}=R_\oplus+alt._{obs}=6375.9km$ is the distance between him and the geocenter. We have thus 

\begin{equation}
v_E = \frac{2\pi r_E}{86400} = 0.46km/s
\end{equation}

Table~\cite{table:planetvalues} gives $v_S < 30.29 km/s$. For planets, we can thus neglect $v_E$ and major $v_{obs}$:

\begin{equation}
v_{obs} < 30.29 km/s
\end{equation}

Now, the maximum error angle by second is the difference between the angle and the angle one second later. It has its maximum when the first angle is 0, so we just need to calculate the angle after one second. For planets, this angle corresponds to Fig~\ref{maxangularerror}.

\begin{figure}
\centering 
\begin{tikzpicture}[scale=2]
\draw [<-] (-2,0) node[anchor=north] {$P_{obs}(t_1)$} -- node[anchor=north] {$\vec{v}_{obs}$} (0,0) node[anchor=north] {$P_{obs}(t_0)$} -- %node[northwest=2mm] {$r_{min}$} 
(0,2) node[anchor=south] {$P_{cb}(t_0)$};
\node [anchor=center] at (-0.25,1.35) {$r_{min}$};
\draw [->] (0,2) -- node[anchor=south] {$\vec{v}_{cb}$} (2,2) node[anchor=south] {$P_{cb}(t_1)$};
\draw [gray] (2,2) -- (-2,0);
\draw (-1.3,0) arc (0:26.56:0.7);
\node [anchor=center] at (-1,0.2) {$\Theta_{max}$};
\end{tikzpicture}
\caption{Maximum angular error for short times}\label{maxangularerror}
\end{figure}

This figure can be easily simplified in a simple triangle and we can deduce $\Theta_{max}$ for planets:

\begin{equation}
\Theta_{max}=atan(\frac{v_{obs}-v_{cb}}{r_{min}})
\end{equation}

For Sun and Moon, the reference is not moving, so we can simply take the same value with $v_{cb}=0$. Table~\cite{table:planetvalues} gives $r_{min}$ and $v_{cb_{max}}$, we can thus deduce Table~\cite{table:thetamaxtime}. Note that the values for the planets are over-pessimistic and could be refined by taking the account that all planets rotate in the same direction. We don't need to do this because the biggest error, which we'll take as reference, is on the Moon which has a quite accurate value.

\begin{table}
\centering
\begin{tabular}{|l|S[table-format=1.1,table-figures-exponent=1]|}
\hline
\multicolumn{1}{|c|}{\textbf{Celestial body}} & \multicolumn{1}{c|}{\textbf{$\Theta_{max}$ (s)}} \\\hline
Sun & 4.3e-2\\\hline
Moon & 6.1e-1\\\hline
Mercury & 2.4e-1\\\hline
Mars & 2.1e-1\\\hline
Venus & 3.6e-1\\\hline
Saturn & 6.9e-3\\\hline
Jupiter & 1.5e-5\\\hline
\end{tabular}
\caption{Maximum angular error for a time error of 1s}
\label{table:thetamaxtime}
\end{table}

\mysection{Astronomical timescales}

\mysubsection{Atomic time (TAI)}

\mysubsection{Earth Rotation Time}

\mysubsection{Dynamical Time}

\mysubsubsection{Based on TAI}

\mysubsubsection{Truely relativistic time}

\mysection{Time used in calendrical calculations}

\mysubsection{Mean Sun?}

\mysection{Conversions}


\mychapter{Coordinate systems}

\mychapter{Errors due to Time and Coordinate system calculations}

\mysection{Delta T uncertainty}

\mysection{Nutation Effect}

\mysection{Obliquity of the Ecliptic}

\mysection{Earth Rotation Angle (ERA)}

\mysection{Position of the observer}

Calendrical calculations are supposedly made by humans above the sea, not at the geocenter. So the observed longitude of a celestial body 

$R_o$ is the distance of the observer from the geocenter ($R_E+altitude$)
$D_{E-M}$ is the Earth-Moon distance
$\Theta$ is the resulting error

The error will be maximized by an observation of the moon when it's $90°$ from vernal equinox and at its perigee (about $D_{E-Mmin}\approx360000km$) made by someone on the top of the Everest ($R_o=R_E+8,848km$) and considering the Everest is on the Ecliptic. In this case we would have:

$$\Theta_{max} = atan(R_o/D_{E-Mmin}) = 3655as$$

% atan(6379.85/360000) = .01771995065419387521rad = 3655s

Which is an error to be avoided by all means!

If we take an observation at sea level, with a $30°$ angle and an average distance (Sun-Earth distance for inner planets and Sun-planet distance for outer ones), we can compute the following table:

\begin{center}
\sisetup{table-figures-exponent = 1}
\begin{tabular}{|l|S|S|}
\hline
\textbf{Celestial body} & \multicolumn{1}{c|}{\textbf{$\Theta_{max}$}} & \multicolumn{1}{c|}{\textbf{$\Theta$}}  \\\hline
Sun & 9 & 4\\\hline % min: 147Mkm, mean: 150km
Moon & 3655 & 1706\\\hline % min: 360kkm, mean: 385kkm
Mercury & 17 & 7\\\hline %  min: 77 Mkm, mean: 100Mkm
Mars & 24 & 3\\\hline % min: 54.6Mm, mean: 225Mkm
Venus & 35 & 4\\\hline % min: 38Mkm, mean: 150Mkm
Saturn & 1 & 453\\\hline % min  1200Mkm, mean: 1450Mkm
Jupiter & 2e-3 & 8e-4 \\\hline % min: 629kMkm, mean: 770kMkm
\end{tabular}
\end{center}

\mychapter{Other inaccuracies}

\mysection{iterations in the light tralvel time}

\mysection{Aberration of light}



\end{document}
