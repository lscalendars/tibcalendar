\documentclass[%
a4paper,% paper size
pagesize,%
12pt,% text size
parskip=off,% inter-paragraph space, automatically sets paragraph indentation
bibliography=totoc,% bibliography in the table or contents
numbers=noenddot,%
DIV=12,% margin settings
twoside=semi,% right and left difference, but same margins
headings=small,% small headings
]{scrartcl}

\usepackage{eroux}

\title{Astronomical calculations in lunisolar calendars}
\subtitle{Overview of accuracy and optimization possibility}
\author{Élie Roux \href{mailto:elie.roux@telecom-bretagne.eu}{<elie.roux@telecom-bretagne.eu>}} 
\date{\today} 

\pagestyle{scrheadings}
\ifoot[]{} 
\cehead[]{\textsc{\rightmark}}
\cohead[]{\textsc{Astronomical calculations in lunisolar calendars}}
% et enfin le numéro de page dans le pied de page extérieur
\ofoot[\oldstylenums\thepage]{\oldstylenums\thepage}
\cfoot[]{}
\ohead[]{}
\ihead[]{}

\begin{document}

\maketitle

\begin{abstract}
blabla $x=y$
\end{abstract}

\tableofcontents
%\cleardoubleevenemptypage
%\clearpage
\newpage

\section{Introduction}

This will be the introduction

\subsection{What it this article?}

\subsection{Genesis of this article}

\subsection{What this article is not}

\subsection{Precision needed in }

\section{The different calculation methods}

\subsection{Variations Séculaires des Orbites Planétaires (VSOP)}

\subsubsection{vsop87}

\subsubsection{vsop2013}

\subsection{ELP/MPP02 and LEA-406}

\subsection{JPL Ephemeris}

%% Timescales

\section{Timescales}

\subsection{Atomic time (TAI)}

\subsection{Earth Rotation Time}

\subsection{Dynamical Time}

\subsubsection{Based on TAI}

\subsubsection{Truely relativistic time}

\subsection{Time used by calendrical calculations}

\subsection{Time used by ephemeris calculations}

\subsection{Time used by calendrical calculations}

\section{Coordinate systems}

\section{Errors due to Time and Coordinate system calculations}

\subsection{Delta T uncertainty}

\subsection{Nutation Effect}

\subsection{Obliquity of the Ecliptic}

\subsection{Earth Rotation Angle (ERA)}

\subsection{Position of the observer}

Calendrical calculations are supposedly made by humans above the sea, not at the geocenter. So the observed longitude of a celestial body 

$R_o$ is the distance of the observer from the geocenter ($R_E+altitude$)
$D_{E-M}$ is the Earth-Moon distance
$\Theta$ is the resulting error

The error will be maximized by an observation of the moon when it's $90°$ from vernal equinox and at its perigee (about $D_{E-Mmin}\approx360000km$) made by someone on the top of the Everest ($R_o=R_E+8,848km$) and considering the Everest is on the Ecliptic. In this case we would have:

$$\Theta_{max} = atan(R_o/D_{E-Mmin}) = 3655as$$

% atan(6379.85/360000) = .01771995065419387521rad = 3655s

Which is an error to be avoided by all means!

If we take an observation at sea level, with a $30°$ angle and an average distance (Sun-Earth distance for inner planets and Sun-planet distance for outer ones), we can compute the following table:

\begin{center}
\sisetup{table-figures-decimal = 1, 
  table-figures-integer = 4,
  table-format=4.1,
  table-space-text-post = \si{mas},
  table-number-alignment = right,
  table-unit-alignment = left,
  table-text-alignment = center,
%  table-align-text-post = false
  }
\begin{tabular}{|l|S|S[table-unit-alignment = left,table-format=4.1]|}
\hline
\textbf{Celestial body} & \textbf{$\Theta_{max}$} & \textbf{$\Theta$}  \\\hline
Sun & \SI{9}{\second} & \SI{4}{\second}\\\hline % min: 147Mkm, mean: 150km
Moon & \SI{3655}{\second} & \SI{1706}{\second}\\\hline % min: 360kkm, mean: 385kkm
Mercury & \SI{17}{\second} & \SI{6.7}{\second}\\\hline %  min: 77 Mkm, mean: 100Mkm
Mars & \SI{24}{\second} & \SI{3}{\second}\\\hline % min: 54.6Mm, mean: 225Mkm
Venus & \SI{35}{\second} & \SI{4}{\second}\\\hline % min: 38Mkm, mean: 150Mkm
Jupiter & \SI{2}{mas} & \SI{0.8}{mas} \\\hline % min: 629kMkm, mean: 770kMkm
Saturn & \SI{1}{\second} & \SI{453}{mas}\\\hline % min  1200Mkm, mean: 1450Mkm
\end{tabular}
\end{center}

\section{Other inexact algorithms}

\subsection{iterations in the light tralvel time}

\end{document}
