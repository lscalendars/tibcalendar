\mychapter{What precision for calendrical calculations?}

\mysection{Trueness and precision}

This is quite important here to differenciate trueness and precision, both being part of accuracy. These terms are officially defined in \cite{VIM}\footnote{in accordance to ISO-5724\cite{ISO5725}} as closeness of agreement between:

\begin{description}
\item[trueness] the average of an infinite number of replicate measured quantity values and a reference quantity value
\item[precision] indications or measured quantity values obtained by replicate measurements on the same or similar objects under specified conditions
\item[accuracy] a measured quantity value and a true quantity value of a measurand
\end{description}

It makes sense to translate them to astronomical calculations, taking a calculation as a \emph{measure}. \ref{ATP} is useful to get a better understanding.

\begin{figure}[h]
\centering
\def\svgwidth{10cm}
\input{ATP.pdf_tex}
\label{ATP}
\caption{Accuracy, Precision and Trueness of calculations}
\end{figure}

\mysection{Precision of different computer methods}

We will take here the 

\mysection{Scales used in lunisolar calendars}

\mysubsection{The tibetan calendar}

\mysubsection{Other calendars}
